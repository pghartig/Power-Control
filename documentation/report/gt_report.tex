\documentclass[12pt,a4paper]{report}
\usepackage[utf8]{inputenc}
\usepackage{amsmath}
\usepackage{amsfonts}
\usepackage{amssymb}
\input defs.tex
\bibliographystyle{alpha}


\title{Game Theoretic Solutions to Power Control in MIMO Communications Systems}
\author{Peter Hartig}

\begin{document}
\maketitle

\begin{abstract}
This work investigate resource allocation strategies for such systems and in particular, those strategies minimizing system overhead. 
\end{abstract}

\newpage
\tableofcontents
\newpage

\chapter{Introduction}
\section{Introduction}
Modern communication systems often incorporate numerous, uncoordinated users competing for a limited resource.
In this work, a game theoretic approach considers a wireless communications network with uncoordinated  macrocell and femtocell users. The goal of this approach is to find Nash Equilibrium in which Femtocell base stations optimize their utility while ensuring macrocell users intercept interference below a given threshold. 
Previous work has shown solutions for systems in which the femtocell base stations transmit over a SISO channel; thus the MIMO case is investigated here. 
One of the primary goals of this work is to achieve Nash Equilibrium via distributed solutions which minimize system overhead. As seen in the following, the most general setup of this game does not permit such solutions and therefore a refined problem is purposed.


\section{Relevant Tools/Theory}

It is useful to first review relevant tools for solving such games. Together, these tools provide a series of steps to reach Nash Equilibrium via a distributed solutions. 

\begin{enumerate}
\item The Concave N-Person game a framework for proving existence and uniqueness of NE for games fulfilling certain criteria.

\begin{itemize}
\item
\textbf{Definition:} A game in which the individual player optimization problems are concave and the strategy set resulting from problem constraints (potentially player coupled) is convex.
\item 
\textbf{Existence of Nash Equilibrium:} Pure Strategy NE exist for Concave N-Person  games \cite[Thm1]{rosen1964existence}. 
\item
\textbf{Uniqueness of Nash Equilibrium:} To prove uniqueness of NE for Concave N-Person  games, the "Normalized Nash Equilibrium" is introduced.
First a weighted sum of the player utility functions is taken.
\begin{equation}
\sigma(u,r)  = \Sigma_{\mathrm{f=1}}^{\mathrm{F}} r_{\mathrm{f}}U_{\mathrm{f}}(), \; 
r_{\mathrm{f}} \geq 0
\end{equation}
 The function $G(b,r) $ is defined as the Jacobian of $g(b,r) $ which is defined as
\begin{enumerate}
\item  
\begin{equation}
g(b,r)= 
\begin{bmatrix}
r_1 \nabla U_{1}(b)
\\
r_2 \nabla U_{2}(b)
\\
\vdots\\
r_F \nabla U_{F}(b)
\end{bmatrix}
\end{equation}

\begin{itemize}
\item
Negative Definiteness of the matrix $[G(b,r)+G^{T}(b,r)] $ is a sufficient condition for Diagonally Strict Concavity of $\sigma(u,r)$ which is a sufficient condition for uniqueness of a NNE \cite[Thm4]{rosen1964existence}

\end{itemize}

\end{enumerate}
\end{itemize}
\item The Potential function of a game.
\begin{itemize}
\item
\textbf{Definition:} A function
$ \Psi()$ which satisfies:
\begin{equation}\label{potential_game_condition}
\frac{\partial \Psi(x)}{\partial u_{f,i}}
 =
 \frac{\partial U_(x)}{\partial u_{f,i}}
\end{equation}
\item Global optima to the Potential function correspond to NE of the game.
\item If a Concave N-Person game game admits a potential function, the potential function will be concave. NE may then be found efficiently by convex optimization tools. 
\end{itemize}



\item Distributed Optimization
\begin{itemize}
\item Distributed solutions minimize the system overhead required in sharing
\begin{itemize}
\item Begin with a central optimization problem.
\item For the dual of the central problem
\item Evaluate if problem can be separated into sub-problems each with independent optimization variables.
\item Perform an ascent of the dual problem by iterating between solving for the dual function of the sub-problems and then broadcasting these solutions to perform the gradient step of the dual problem. 

\end{itemize} information between network users. 
\end{itemize}
\item Concave Programming?

\end{enumerate}
\subsection{Using the tools}
(May want to make a flow chart?)
The typical usage of these tools is:
first verify existence and uniqueness for a Normalized NE of the game using n-concave games,
second find a centralized version n-concave game which will be a concave function over a concave set,
find a distributed solution to the single convex problem resulting from the potential funciton.


\section{Outline}
Look at how the tools described above will be used in the work. 
\begin{itemize}
\item 
\ref{genmodel} Introduces the system model for the most general setup of the problem.
\item 
\ref{genproblem} Introduces various tools for solving for Nash Equilibrium and analyzes the general setup for compatibility with these tools. 
\item
\ref{conmodel} Provides refinements to the system model in order to allow for ...
\item 
\ref{conproblem} Details solving the full problem ...
\item 
details numerical comparisons of the two proposed methods for identifying Nash Equilibrium of the game. 
\end{itemize}
\chapter{System Model}

\section{General System Model}\label{genmodel}

\subsection{Players: Femtocell Base Stations}


Each players of the game, Individual Femtocell base stations $f \in \{1 ... F\}$ (FC-BS), is characterized by:
\begin{itemize}
\item 
$T_f$ antennas with which to transmit to $K_f$ femtocell users (it is assumed that $T_f \geq K_f$).
\\
\item 
	Precoding matrix $\mathbf{U}_{\mathrm{f}} \in \mathbb{C}_{T_f \times K_f}$ such that the transmitted 		
	signal is $\mathbf{s}_{\mathrm{f}
	}= \mathbf{U_{\mathrm{f}}}\mathbf{x_{\mathrm{f}}}$. Here $\mathbf{x_{\mathrm{f}}}$ is the 		
	vector of symbols for users of FC-BS $f$ normalized such that $E[\|\mathbf{x}_{\mathrm{f,i}}
	\|_2^2]=1$ and $E[\mathbf{x}_{\mathrm{f}}\mathbf{x}_{\mathrm{f}}^H]=\mathbf{I}$.
\\
\item 
	Power constraint $trace(\mathbf{U}_f^H\mathbf{U}_f) \leq P^{Total}_{f} $.

\item 
	Cost function $U_f() =
	\sum_{\mathrm{i=1}}^{\mathrm{K_f}}
    	 U_{\mathrm{f,i}}(\gamma_{\mathrm{f,i}}) $
    	in which all $U_{\mathrm{f,i}}()$ are non-decreasing in argument. (No assumption about concavity yet)

\item 
	Knowledge of the downlink channel matrix $\mathbf{H_\mathrm{f}} \in \mathbb{C}_{K_f \times T_f} $ to its $K_f$ users.
% TODO(Simulate degradation with incomplete CSI solution?)
\\
\item
	 FC-BSs are assumed to be spaced far apart in distance such that FC-BS $f$ 
	 causes no interference to the users of any other FC-BS.
\end{itemize}

\subsection{Macrocell Users}
Macrocell users $m \in \{1 ... M\}$ introduce constraints into this game. These users are characterized by:

\begin{itemize}
\item 
	Received interference constraint
	$\sum^F_{f=1} \mathbf{\tilde{h}}_{\mathrm{m,f}}^T  \mathbf{U_{\mathrm{f}}} 						
	\mathbf{U_{\mathrm{f}}^{\mathrm{H}}} \mathbf{\tilde{h}_{\mathrm{m,f}}^*} \leq I^{Threshold}		
	_{\mathrm{m}} $.

\item 
	FC-BS $f$ is assumed to know the downlink channel matrix $\tilde{\mathbf{H}_{\mathrm{f}}} \in \mathbb{C}_{M \times T_f}$ to all $M$ Macrocells users.
\\
\end{itemize}

\subsection{Femtocell Users}
\begin{itemize}

\item User $i$ of FC-BS $f$ has signal to interference plus noise ratio (SINR):
	\begin{equation*}
	\gamma_{\mathrm{f,i}} = \frac{\|\mathbf{h^H_{\mathrm{f,i}}u_{\mathrm{f,i}}}\|^2}
	{\sigma^2_{noise}   +
	\underbrace{
	 \sum_{\mathrm{\tilde{f}}=1,\mathrm{\tilde{f}}\neq f}^{\mathrm{F}} \sum_{\mathrm{u=1}}^{K_{\mathrm{\tilde{f}}}}
	\|\mathbf{h^H_{\mathrm{\tilde{f},u}}u_{\mathrm{\tilde{f},i}}}\|^2}_{\mathrm{inter-cell}}
	 + 
	 \underbrace{
	 \sum_{\mathrm{\tilde{k}\neq i}}^{\mathrm{K_f}}
	 \|\mathbf{h^H_{\mathrm{f,\tilde{k}}}u_{\mathrm{f,\tilde{k}}}}\|^2}_{\mathrm{intra-cell}}}
	  \; \mathrm{i \in \{1 ... K_f\}}\end{equation*}
\\
with AWGN $\sim \mathcal{CN}(0,\sigma^2_n)$
\\

Assuming negligible inter-cell interference, this reduces to
	\begin{equation*}
	\gamma_{\mathrm{f,i}} = \frac{\|\mathbf{h^H_{\mathrm{f,i}}u_{\mathrm{f,i}}}\|^2}
	{\sigma^2_{noise} 
	 + \sum_{\mathrm{\tilde{k}\neq i}}^{\mathrm{K_f}}
	  \|\mathbf{h^H_{\mathrm{f,\tilde{k}}}u_{\mathrm{f,\tilde{k}}}}\|^2}
	  \; \mathrm{i \in \{1 ... K_f\}}
	\end{equation*}
\\


\end{itemize}





\subsection{General Optimization Problem}

Each player $f$ attempts to maximize utility function $U_f()$ while satisfying the interference constraints imposed by the macro cell users.
\par




	\begin{subequations}
	\label{optim}
	\begin{align}
	    \underset{\mathbf{U}_{\mathrm{f}} }{\text{argmin}} \;
	    & - \sum_{\mathrm{i=1}}^{\mathrm{K_f}}
    	U_{\mathrm{f,i}}(\gamma_{\mathrm{f,i}}) \label{player_opt} \\
	    \text{subject to} \; &
	   \sum^F_{f=1} \mathbf{\tilde{h}}_{\mathrm{m,f}}^T  \mathbf{U_{\mathrm{f}}}		
	\mathbf{U_{\mathrm{f}}^{\mathrm{H}}} \mathbf{\tilde{h}_{\mathrm{m,f}}^*} \leq I^{Threshold}		
	_{\mathrm{m}} & m \in \{1 ...M\} 
		\label{interference_const_gen}\\
        & trace(\mathbf{U_f^H}\mathbf{U_f}) \leq P^{Total}_{f} \label{power_const_gen}\\
        & \langle \mathbf{h_{f,j}}\mathbf{u_{f,i}} \rangle =0\ & \; \forall j \in \{1... K_f\}\backslash i ,\; \forall i \in \{1 ... K_f\} \label{zf_const_gen}
	\end{align}
	\end{subequations}
	
Note that in order to ensure the game admits a potential function, $\mathbf{U}_f$ is restricted to the set of zero-forcing matrices by constraint \eqref{zf_const_gen}.	
	
\subsection{Concave N-Person game analysis of general setup}

Sufficient conditions for a concave problem are:

\begin{enumerate}
\item The utility function is concave in its argument.
\begin{itemize}
\item 
Due to constraint \eqref{zf_const_gen}, $\gamma_{\mathrm{f,i}}$ takes the form
	\begin{equation}\label{zf_snr}
	\gamma_{\mathrm{f,i}} = \frac{\|\mathbf{h^H_{\mathrm{f,i}}u_{\mathrm{f,i}}}\|^2}
	{\sigma^2_{noise}  
	}
	= 
	\frac{\mathbf{u^H_{\mathrm{f,i}}h_{\mathrm{f,i}}h^H_{\mathrm{f,i}}u_{\mathrm{f,i}}}}
	{\sigma^2_{noise}  
	}
	\end{equation}
	As $\mathbf{h}_{\mathrm{f,i}}\mathbf{h}^H_{\mathrm{f,i}}$ is positive semi-definite, $\gamma_{\mathrm{f,i}}$ is convex in ${\mathbf{u}_{\mathrm{f,i}}}$. 
\item
The composition $U_{\mathrm{f,i}}(\gamma_{\mathrm{f,i}}) $ is concave only if $U_{\mathrm{f,i}}() $ is concave and non-increasing; violating the non-decreasing definition of $U_{\mathrm{f,i}}() $.
\end{itemize}

\item
The constraints form a convex, closed and bounded set. 

\begin{itemize}

\item
	Constaint \eqref{interference_const_gen} contains $M$ quadratic constraints on $\mathbf{U_f}$ and 
	can be rewritten as 

\begin{gather*}
	\sum_{f=1}^F
	trace(\mathbf{U_f^H} \mathbf{\tilde{h}_{m,f}} \mathbf{\tilde{h}_{m,f}^H} \mathbf{U_f} )\leq 
	I^{Threshold}_{m}.
\end{gather*}
This can be further decomposed into  
	\begin{gather*}
	\sum_{f=1}^F
	\sum_{i=1}^{f_i}
	\mathbf{u_{\mathrm{f,i}}^H}\mathbf{\tilde{h}_{\mathrm{m,f}}} \mathbf{\tilde{h}}_{\mathrm{m,f}}^H
	\mathbf{u_{\mathrm{f,i}}} \leq I^{Threshold}_{m}
	\end{gather*}
in which the term $ \mathbf{\tilde{h}_{\mathrm{m,f}}} \mathbf{\tilde{h}}_{\mathrm{m,f}}^H$ is  a positive semi-definite matrix and is, therefore, a convex set 
\cite[p.8,9]{BoV:04}. 
%This is essentially high dimensional ellipsoid.


\item \
	Constraint \eqref{power_const_gen} can be similarly decomposed into the sum
	\begin{gather*}
		\sum_{i=1}^{K_f}\mathbf{u_{\mathrm{f,i}}^{\mathrm{H}}} \mathbf{I} 		
		\mathbf{u_{\mathrm{f,i}}} \leq  P^{Total}_{f}
	\end{gather*}
	in which $\mathbf{I}$ is positive definite and 			
	therefore the constraint is strictly convex for the same 		
	reason as \eqref{interference_const_gen}.
\end{itemize}

\item 
	Constaint \eqref{zf_const_gen} is an affine constraint. 
		\begin{gather*}
		\langle \mathbf{h_{\mathrm{f,j}}}\mathbf{u_{\mathrm{f,i}}} \rangle =0
		\end{gather*}
%Note that affine constaints to not have to satisfy Slater's condition

\end{enumerate}

\subsection{Potential Game for General Problem}
Under the current system model, the game is not a Concave N-Person game. Nevertheless, the game may still admit a potential function which would allow 

TODO: 
\begin{itemize}
\item Need to prove that pure strategy NE would exist in this setting.
\item Explain why even through a potential game can be formed, it may not be useful/feasible to find NE. 
\end{itemize}

\section{Concave System Model}\label{conmodel}

The general setup of this game  seen in the previous section prohibits the use of tools used to find NE efficiently in this context. 
The following adaption of the general problem permits a unique Nash Equilibrium to be found via a single convex problem which may be distributed across players of the game.
\subsection{Players: Femtocell Base Stations}


Individual Femtocell base stations (FC-BS) are the players of this game.
\\
Femtocells are characterized by the following parameters
\begin{itemize}
\item 
	FC-BSs with multiple antennas ($T_f >=1$) can beamform their transmission using the precoding 	
	matrix $\mathbf{U}_{\mathrm{f}} \in \mathbb{C}_{T_f \times K_f}$ .
	The columns of $\mathbf{U}_{\mathrm{f}}$ are normalized such that 
	 $\|\mathbf{u}_{\mathrm{fi}}\|^2 =1 \;\forall i \in \{1 ... K_f\}$.
	 $\mathbf{x_{\mathrm{f}}}$ is the 		
	normalized information symbol vector of FC-BS $f$ such that $E[\|\mathbf{x}_{\mathrm{f}}
	\|_2^2]=1$ and $E[\mathbf{x}_{\mathrm{f}}\mathbf{x}_{\mathrm{f}}^H]=\mathbf{I}_{K_f \times K_f}$ $\forall f \in \{1 ... F\}$.
\\

\item 
$\mathbf{U}_f$ is the normalized, Moore-Penrose inverse to $\mathbf{H_\mathrm{f}}$.
Such that
\begin{gather*}
\langle \mathbf{h_{f,j}}\mathbf{u_{f,i}} \rangle =0\  \; \forall j \in \{1... K_f\}\backslash i ,\; \forall i \in \{1 ... K_f\}
\end{gather*}

\item  
	FC-BS $f$ allocates  transmission power using the diagonal, power allocation  	
	matrix $\mathrm{diag}(\mathbf{p}_{\mathrm{f}})$ with $p_{\mathrm{fi}} \geq 0, \forall i \in \{1 ... K_f\}$
such that the transmitted 		
	signal is 
	$\mathbf{s}_{\mathrm{f}	}= \mathbf{U_{\mathrm{f}}} 
	\mathrm{diag}(\mathbf{p}_{\mathrm{f}})^{\frac{1}{2}}
	\mathbf{x_{\mathrm{f}}}$.
\\
\item 
	FC-BS $f$ has power constraint:
	\begin{gather*}
	trace(E[\mathbf{s}_\mathrm{f}\mathbf{s}_\mathrm{f}^H]) =
	 trace(\mathbf{U_{\mathrm{f}}} 
	\mathrm{diag}(\mathbf{p}_{\mathrm{f}}^{\frac{1}{2}})
	E[\mathbf{x_{\mathrm{f}}}
	\mathbf{x_{\mathrm{f}}^H}]
	\mathrm{diag}(\mathbf{p}_{\mathrm{f}}^{\frac{1}{2}})
	\mathbf{U_{\mathrm{f}}}^H 
	)
	  \leq P^{Total}_{f} 
	  	\end{gather*}

% TODO(Simulate degradation with incomplete CSI solution?)


\item 
	Cost function $U_f() =
	\sum_{\mathrm{i=1}}^{\mathrm{K_f}}
    	U_{\mathrm{f,i}}(\gamma_{\mathrm{f,i}}) $
    	in which all $U_{\mathrm{f,i}}()$ are non-decreasing and concave in argument.

\end{itemize}

\subsection{Macro Cell Users}
Same as the general setup.

\subsection{Femtocell Users}
\begin{itemize}


%\item TODO Decide if there should be minimum rate constraints for Femtocell users in case (Look back at later in case some users are beam-formed out of the transmission). Check if this constraint will disrupt solution.
%\\

\item User $i$ of FC-BS $f$ has SINR:

	\begin{equation*}
	\gamma_{\mathrm{f,i}} = 
	\frac{p_{\mathrm{fi}}\|\mathbf{h^H_{\mathrm{f,i}}u_{\mathrm{f,i}}}x_{\mathrm{fi}}\|^2}
	{\sigma^2_{noise}   +
	\underbrace{
	\sum_{\mathrm{\tilde{f}=1,\tilde{f} \neq f}}^{\mathrm{F}}
	\sum_{\mathrm{\tilde{k}\neq i}}^{\mathrm{K_f}}
	  p_{\mathrm{f\tilde{k}}}\|\mathbf{h^H_{\mathrm{f,\tilde{k}}}u_{\mathrm{f,\tilde{k}}}}x_{\mathrm{f\tilde{k}}}\|^2}_
	  {\mathrm{Inter-cell}}+ \underbrace{
	\sum_{\mathrm{\tilde{k}\neq i}}^{\mathrm{K_f}}
	  p_{\mathrm{f\tilde{k}}}\|\mathbf{h^H_{\mathrm{f,\tilde{k}}}u_{\mathrm{f,\tilde{k}}}}x_{\mathrm{f\tilde{k}}}\|^2}
	 _{\mathrm{Intra-cell}}}
	  \; \mathrm{i \in \{1 ... K_f\}}\end{equation*}
\\
with AWGN $\sim \mathcal{N}(0,\sigma^2_n)$
\\

Assuming negligible inter-cell interference, this reduces to
	\begin{equation*}
	\gamma_{\mathrm{f,i}} = \frac{p_{\mathrm{fi}}\|\mathbf{h^H_{\mathrm{f,i}}
	u_{\mathrm{f,i}}}x_{\mathrm{fi}}\|^2}
	{\sigma^2_{noise} 
	 + \sum_{\mathrm{\tilde{k}\neq i}}^{\mathrm{K_f}}
	  p_{\mathrm{f\tilde{k}}}\|\mathbf{h^H_{\mathrm{f,\tilde{k}}}u_{\mathrm{f,\tilde{k}}}}x_{\mathrm{f\tilde{k}}}\|^2}
	  \; \mathrm{i \in \{1 ... K_f\}}
	\end{equation*}
\\

%This further simplifies assuming that users use a zero-forcing beam-former
%
%\begin{equation}\label{zf_snr}
%\gamma_{\mathrm{f,i}} = \frac{|\mathbf{h^H_{\mathrm{f,i}}u_{\mathrm{f,i}}}|^2}
%{\sigma^2_{noise}  
%}
%\end{equation}
%\\

\end{itemize}


\subsection{Optimization Problem of player $f$}


	\begin{subequations}
	\label{optim}
	\begin{align}
	    \underset{\mathbf{p}_{\mathrm{f}} }{\text{minimize}} \;
	    & - \sum_{\mathrm{i=1}}^{\mathrm{K_f}}
    	U_{\mathrm{f,i}}(p_{\mathrm{f,i}}) \label{player_opt_c} \\
	    \text{subject to} \; &
	  \sum^F_{f=1} \mathbf{\tilde{h}}_{\mathrm{m,f}}^T  \mathbf{s}_{\mathrm{f}} 						
	\mathbf{s_{\mathrm{f}}^{\mathrm{H}}} \mathbf{\tilde{h}_{\mathrm{m,f}}^*} \leq I^{Threshold}		
	_{\mathrm{m}} & m \in \{1 ...M\} 
		\label{interference_const_c}\\
        & trace(\mathbf{s}_\mathrm{f}\mathbf{s}_\mathrm{f}^H)  \leq P^{Total}_{f}  \label{power_const_c}\\
        & p_{\mathrm{f,i}} \geq 0 &  i \in \{1 ...K_{\mathrm{f}}\} \label{pos_power_const_c}
	\end{align}
	\end{subequations}


\subsection{Concave N-Person game analysis of concave setup}

Sufficient conditions for a concave problem are:

\begin{enumerate}
\item The utility function is assumed concave in its argument $p_{\mathrm{f,i}}$.

\item
The constraints form a convex, closed and bounded set. 

\begin{itemize}

\item
	Constaint \eqref{interference_const_c} contains $M$ affine constraints on $diag(\mathbf{p_{\mathrm{f}}})$ and 
	can be rewritten as 

\begin{gather*}
	  \sum^F_{f=1} \mathbf{\tilde{h}}_{\mathrm{m,f}}^T  \mathbf{s}_{\mathrm{f}} 						
	\mathbf{s_{\mathrm{f}}^{\mathrm{H}}} \mathbf{\tilde{h}_{\mathrm{m,f}}^*} = 
	\sum^F_{f=1} trace(\mathbf{\tilde{h}}_{\mathrm{m,f}}^T U_{\mathrm{f}}diag(\mathbf{p_{\mathrm{f}}})^{\frac{1}{2}}\mathbf{x_{\mathrm{f}}}
	 \mathbf{x_{\mathrm{f}}}^H  diag(\mathbf{p_{\mathrm{f}}})^{\frac{1}{2}} U_{\mathrm{f}}^H
	 \mathbf{\tilde{h}}_{\mathrm{m,f}}^*
	)	
	\leq I^{Threshold}_{\mathrm{m}} 
\end{gather*}

\item \
	Constraint \eqref{power_const_c} is  afine in $diag(\mathbf{p_{\mathrm{f}}})$.
	
\item \
	Constraint \eqref{pos_power_const_c} is affine in $diag(\mathbf{p_{\mathrm{f}}})$.
\end{itemize}

%Note that affine constaints to not have to satisfy Slater's condition

\end{enumerate}

This problem satisfies the conditions for the Concave N-Persion game and therefore a pure strategy Nash Equilibrium exists due to 
\cite[Thm1]{rosen1964existence}.

Next, the conditions for unqiueness of this NE are verified. As the utility function $U_{\mathrm{fi}}(p_{\mathrm{fi}})$ is concave in argument and independent of all $p_{\mathrm{fj}}$ the matrix $[G(b,r)+G^{T}(b,r)] $ will be diagonal with (strictly) negative values.
\subsection{Potential Game for Concave Problem}

\textbf{Potential Function:} The proposed potential function satisfying \eqref{potential_game_condition} is 

\begin{gather*} \label{Potential_Function}
\Psi(\mathbf{p}) = \sum_{f = 1}^{F} U_f(\mathbf{p_{\mathrm{f}}}) 
\end{gather*}

resulting in the single optimization problem
	
		\begin{subequations}
	\label{optim}
	\begin{align}
	    \underset{\mathbf{p}}{\text{minimize}}
	    & \; \Psi(\mathbf{p}) \label{potential_game} \\
	    \text{subject to} \; &
	  \sum^F_{f=1} \mathbf{\tilde{h}}_{\mathrm{m,f}}^T  \mathbf{s}_{\mathrm{f}} 						
	\mathbf{s_{\mathrm{f}}^{\mathrm{H}}} \mathbf{\tilde{h}_{\mathrm{m,f}}^*} \leq I^{Threshold}		
	_{\mathrm{m}} & m \in \{1 ...M\} 
		\label{interference_const}\\
        & trace(\mathbf{s}_\mathrm{f}\mathbf{s}_\mathrm{f}^H)  \leq P^{Total}_{f}  \label{power_const}
        & \forall f \in \{1 ... F\}\\
        & p_{\mathrm{f,i}} \geq 0 &  \forall i \in \{1 ...K_{\mathrm{f}}\} \; \forall f \in \{1 ... F\}\label{pos_power_const}
	\end{align}
	\end{subequations}

%\begin{theorem}\label{distributed}
%\cite{ghosh2015normalized}
%If a game's potential function is strictly concave and the derivative of the function with respect to the individual players variables are independent of the other player variables, then there exists a distributed solution.
%\end{theorem}

\subsection{Distributed Solution to the Game}
A distribtued solution to \eqref{potential_game} is desired. This may allow for minimal communication overhead between processes (in this case players and macro users).
\subsubsection{Central Problem Resulting from Potential Game}
As described in \cite[p.~8,9]{boyd2011distributed}, the dual ascent method can be used to find a distributed solution to this problem using the Lagrangian of \eqref{potential_game}. 
\\
\begin{multline}
L(\mathbf{U,\lambda}) = 
\;
\sum_{f=1}^F U_f() 
+
\sum_{\mathrm{m=1}}^M \lambda_{\mathrm{m}}
(	  \sum^F_{f=1} \mathbf{\tilde{h}}_{\mathrm{m,f}}^T  \mathbf{s}_{\mathrm{f}} 						
	\mathbf{s_{\mathrm{f}}^{\mathrm{H}}} \mathbf{\tilde{h}_{\mathrm{m,f}}^*} - I^{Threshold}		
	_{\mathrm{m}} )
\\
+ 
\sum_{f=1}^F
\chi_{\mathrm{f}}(trace(\mathbf{s}_\mathrm{f}\mathbf{s}_\mathrm{f}^H)-P^{Total}_{f} )
+
\sum_{f=1}^F 
\sum_{i=1}^{K_f}
\nu_{\mathrm{f,i}}(-p_{\mathrm{f,i}})
\end{multline}

The corresponding dual function and dual problem are then 
\begin{gather*}
g(\lambda,\nu) = \underset{\mathbf{U}}{\mathrm{argmin}}\;L(\mathbf{U,\lambda})
\end{gather*}
\begin{gather*}
g(\lambda,\nu) = \underset{\lambda}{\mathrm{argmax}}\;\underset{\mathbf{U}}{\mathrm{argmin}}\;L(\mathbf{U,\lambda})
\end{gather*}
.



This dual function can then be decomposed into F component functions
\begin{multline}
g_f(\lambda,\nu) = \underset{\mathbf{p_f}}{\mathrm{argmin}}
\{
\;
U_f() 
+
\sum_{\mathrm{m=1}}^M \lambda_{\mathrm{m}}
(\mathbf{\tilde{h}}_{\mathrm{m,f}}^T  \mathbf{s}_{\mathrm{f}} 						
	\mathbf{s_{\mathrm{f}}^{\mathrm{H}}} \mathbf{\tilde{h}_{\mathrm{m,f}}^*} - \frac{I^{Threshold}_{\mathrm{m}}}{F})
\\
+ 
\chi_{\mathrm{f}}(trace(\mathbf{s}_\mathrm{f}\mathbf{s}_\mathrm{f}^H)-P^{Total}_{f} )
+
\sum_{i=1}^{K_f}
\nu_{\mathrm{f,i}}(-p_{\mathrm{f,i}})\}
\end{multline}
\\

The following steps can then be iterated in order to reach an optimal solution. 
\begin{enumerate}
\item 
Individual players can solve $ g_f(\lambda,\nu) $ independently.
\begin{itemize}
\item Setting $\frac{\partial g_f(\lambda,\nu)}{\partial p_{\mathrm{f,i}}} = 0$ 

\item 
Solving for $p_{\mathrm{fi}}$ using \eqref{zf_snr_expanded} yields

\begin{gather}
p_{\mathrm{fi}} = (\sum_{\mathrm{m=1}}^{\mathrm{M}}\lambda_{\mathrm{m}}\|\mathbf{\tilde{h}}_{m,f}^T \mathbf{U_f}_{\mathrm{i}}\|^2_2
+\chi_{f} \|\mathbf{u}_{\mathrm{f,i}}\|^2_2
-\nu_{\mathrm{f,i}}
 )^{-1}
  - \sigma^2_n
\end{gather}

\end{itemize}
\item 
Using $g(\lambda,\nu) = \sum_{f=1}^{F}g_f(\lambda,\nu)$ and the calculus of subgradients $\partial g(\lambda,\nu) = \sum_{f=1}^{F} \partial g_f(\lambda,\nu)$, the dual variables can updated by 

\begin{gather}
\lambda_{\mathrm{m}}^{\mathrm{k+1}} = 
\lambda_{\mathrm{m}}^{\mathrm{k}}
+
\alpha^{\mathrm{k}}*
(
\sum _{\mathrm{f=1}}^{\mathrm{F}}
\sum _{\mathrm{i=1}}^{\mathrm{K_{\mathrm{f}}}}
p_{\mathrm{fi}}
\|\mathbf{\tilde{h}}_{\mathrm{mf}}^T \mathbf{u_{\mathrm{fi}}}\|^2_2 
- I_{\mathrm{m}}
)
\end{gather}


\begin{gather}
\chi_{\mathrm{f}}^{\mathrm{k+1}} = 
\chi_{\mathrm{f}}^{\mathrm{k}}
+
\alpha^{\mathrm{k}}*
(\sum_{\mathrm{i=1}}^{\mathrm{K_{\mathrm{f}}}} p_{\mathrm{fi}} - P_{\mathrm{f}}^{Total}) 
\end{gather}

\begin{gather}
\nu_{\mathrm{fi}}^{\mathrm{k+1}} = 
\nu_{\mathrm{fi}}^{\mathrm{k}}
+
\alpha^{\mathrm{k}}*
(-p_{\mathrm{fi}})
\end{gather}

using predefined $\alpha^{\mathrm{k}}$ which must satisfy certain sumability conditions.



\end{enumerate} 

TODO 
\begin{itemize}
\item Detail the required information passing between system components needed for such a solution
\end{itemize}

\chapter{Results}

\chapter{Conclusion}

\newpage
\bibliography{gt_report}
\end{document}
