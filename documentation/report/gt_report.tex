\documentclass[12pt,a4paper]{report}
\usepackage[utf8]{inputenc}
\usepackage{amsmath}
\usepackage{amsfonts}
\usepackage{amssymb}
\input defs.tex
\bibliographystyle{alpha}


\title{Game Theoretic Solutions to Power Control in MIMO Communications Systems}
\author{Peter Hartig}

\begin{document}
\maketitle

\begin{abstract}
Modern communication systems often incorporate numerous, uncoordinated users with MIMO communication links competing for a limited resource. This work investigate resource allocation strategies for such systems and in particular, those strategies minimizing system overhead. 
\end{abstract}

\newpage
\tableofcontents
\newpage

\chapter{Introduction}
This game considers a wireless communications network which includes uncoordinated  macro cell and Femtocell users. Femtocell base stations must ensure that macro cells users intercept an amount of interference below a threshold. The strategy for transmission across all such Femtocell base stations in a network is investigated here in a game theoretic context. 
One of the primary goals of this work is to achieve Nash Equilibrium via distributed solutions. As seen in the following, the most general setup of this game does not permit such solutions and therefore a refined problem is purposed.


\chapter{System Model}

\section{General System Model}

\subsection{Players: Femtocell Base Stations}


Individual Femtocell base stations (FC-BS) are the players of this game and are characterized by the following:
\begin{itemize}
\item 
	Each FC-BS  $f \in \{1 ... F\}$ is considered to have a number of antennas $T_f$ with which to transmit to $K_f$ Femtocell users. It is assumed throughout the remainder that $T_f \geq K_f$.
\\
\item 
	FC-BSs with multiple antennas ($T_f >=1$) can beamform their transmission using the precoding 	
	matrix $\mathbf{U}_{\mathrm{f}} \in \mathbb{C}_{T_f \times K_f}$ such that the transmitted 		
	signal is $\mathbf{s}_{\mathrm{f}
	}= \mathbf{U_{\mathrm{f}}}\mathbf{x_{\mathrm{f}}}$. Here $\mathbf{x_{\mathrm{f}}}$ is the 		
	vector of symbols for users of FC-BS $f$ normalized such that $E[\|\mathbf{x}_{\mathrm{f}}
	\|_2^2]=1$ and $E[\mathbf{x}_{\mathrm{f}}\mathbf{x}_{\mathrm{f}}^H]=\mathbf{I}$ $\forall f \in \{1 ... F\}$.
\\
\item 
	FC-BS $f$ has power constraint $trace(\mathbf{U}_f^H\mathbf{U}_f) \leq P^{Total}_{f} $.
\\
\item
	 FC-BSs are assumed to be spaced far apart in distance such that FC-BS $f$ 
	 causes no interference to the users of any other FC-BS.
\item 
	FC-BSs have a cost function $U_f() =
	\sum_{\mathrm{i=1}}^{\mathrm{K_f}}
    	U_{\mathrm{f,i}}() $
    	in which all $U_{\mathrm{f,i}}()$ are strictly concave, non-decreasing in argument.

\item 
	FC-BS $f$ is assumed to know the downlink channel matrix $\mathbf{H_\mathrm{f}}$ to its $K_f$ users.
% TODO(Simulate degradation with incomplete CSI solution?)
\\
\end{itemize}

\subsection{Macrocell Users}

\begin{itemize}
\item 
	Macrocell user $m \in \{1 ... M\}$ receives interference, due to
	FC-	BSs, which may not exceed
	$\sum^F_{f=1} \mathbf{\tilde{h}}_{\mathrm{m,f}}^T  \mathbf{U_{\mathrm{f}}} 						
	\mathbf{U_{\mathrm{f}}^{\mathrm{H}}} \mathbf{\tilde{h}_{\mathrm{m,f}}^*} \leq I^{Threshold}		
	_{\mathrm{m}} $.

\item 
	FC-BS $f$ is assumed to know the downlink channel matrix ($\tilde{\mathbf{H}_{\mathrm{f}}}$) from its $T_f$
	transmission antennas to all $M$ Macrocells users.
\\
\end{itemize}

\subsection{Femtocell Users}
\begin{itemize}


%\item TODO Decide if there should be minimum rate constraints for Femtocell users in case (Look back at later in case some users are beam-formed out of the transmission). Check if this constraint will disrupt solution.
%\\

\item User $i$ of FC-BS $f$ has SINR:
	\begin{equation*}
	\gamma_{\mathrm{f,i}} = \frac{\|\mathbf{h^H_{\mathrm{f,i}}u_{\mathrm{f,i}}}\|^2}
	{\sigma^2_{noise}   +
	\underbrace{
	 \sum_{\mathrm{\tilde{f}}\neq f} \sum_{\mathrm{u=1}}^{K_{\mathrm{\tilde{f}}}}
	\|\mathbf{h^H_{\mathrm{\tilde{f},u}}u_{\mathrm{f,i}}}\|^2}_{\mathrm{inter-cell}}
	 + 
	 \underbrace{
	 \sum_{\mathrm{\tilde{k}\neq i}}^{\mathrm{K_f}}
	 \|\mathbf{h^H_{\mathrm{f,\tilde{k}}}u_{\mathrm{f,\tilde{k}}}}\|^2}_{\mathrm{intra-cell}}}
	  \; \mathrm{i \in \{1 ... K_f\}}\end{equation*}
\\
with AWGN $\sim \mathcal{N}(0,\sigma^2_n)$
\\

Assuming negligible inter-cell interference, this reduces to
	\begin{equation*}
	\gamma_{\mathrm{f,i}} = \frac{\|\mathbf{h^H_{\mathrm{f,i}}u_{\mathrm{f,i}}}\|^2}
	{\sigma^2_{noise} 
	 + \sum_{\mathrm{\tilde{k}\neq i}}^{\mathrm{K_f}}
	  \|\mathbf{h^H_{\mathrm{f,\tilde{k}}}u_{\mathrm{f,\tilde{k}}}}\|^2}
	  \; \mathrm{i \in \{1 ... K_f\}}
	\end{equation*}
\\


\end{itemize}
\subsection{General Optimization Problem}

Each player $f$ attempts to maximize utility function $U_f()$ while satisfying the interference constraints imposed by the macro cell users.
\par

If intra-cell interference is  prohibited by the restriction of $\mathbf{U}_f$ to the set of zero-forcing matrices, the optimization problem of player $f$ is:


	\begin{subequations}
	\label{optim}
	\begin{align}
	    \underset{\mathbf{U}_{\mathrm{f}} }{\text{argmin}} \;
	    & - \sum_{\mathrm{i=1}}^{\mathrm{K_f}}
    	U_{\mathrm{f,i}}(\gamma_{\mathrm{f,i}}) \label{player_opt} \\
	    \text{subject to} \; &
	   \sum^F_{f=1} \mathbf{\tilde{h}}_{\mathrm{m,f}}^T  \mathbf{U_{\mathrm{f}}}		
	\mathbf{U_{\mathrm{f}}^{\mathrm{H}}} \mathbf{\tilde{h}_{\mathrm{m,f}}^*} \leq I^{Threshold}		
	_{\mathrm{m}} & m \in \{1 ...M\} 
		\label{interference_const_gen}\\
        & trace(\mathbf{U_f^H}\mathbf{U_f}) \leq P^{Total}_{f} \label{power_const_gen}\\
        & \langle \mathbf{h_{f,j}}\mathbf{u_{f,i}} \rangle =0\ & \; \forall j \in \{1... K_f\}\backslash i ,\; \forall i \in \{1 ... K_f\} \label{zf_const_gen}
	\end{align}
	\end{subequations}


\section{Solving the General Problem}

\subsection{Convexity Analysis of Player Optimization Problem}

Sufficient conditions for a concave problem are:

\begin{enumerate}
\item The utility function is concave in its argument.
\begin{itemize}
\item 
Constraint \eqref{zf_const_gen} implies $\gamma_{\mathrm{f,i}}$ to take the form
	\begin{equation}\label{zf_snr}
	\gamma_{\mathrm{f,i}} = \frac{\|\mathbf{h^H_{\mathrm{f,i}}u_{\mathrm{f,i}}}\|^2}
	{\sigma^2_{noise}  
	}
	= 
	\frac{\mathbf{u^H_{\mathrm{f,i}}h_{\mathrm{f,i}}h^H_{\mathrm{f,i}}u_{\mathrm{f,i}}}}
	{\sigma^2_{noise}  
	}
	\end{equation}
and is therefore $\gamma_{\mathrm{f,i}}$ is convex in ${\mathbf{u}_{\mathrm{f,i}}}$. 
\item
As $U_{\mathrm{f,i}}() $ is strictly concave by definition.
The resulting composition $U_{\mathrm{f,i}}(\gamma_{\mathrm{f,i}}) $ is generally (grammar? ) non-convex.
\end{itemize}

\item
The constraints form a convex, closed and bounded set. 

\begin{itemize}

\item
	Constaint \eqref{interference_const_gen} contains $M$ quadratic constraints on $\mathbf{U_f}$ and 
	can be rewritten as 

\begin{gather*}
	\sum_{f=1}^F
	trace(\mathbf{U_f^H} \mathbf{\tilde{h}_{m,f}} \mathbf{\tilde{h}_{m,f}^H} \mathbf{U_f} )\leq 
	I^{Threshold}_{m}.
\end{gather*}
This can be further decomposed into  
	\begin{gather*}
	\sum_{f=1}^F
	\sum_{i=1}^{f_i}
	\mathbf{u_{\mathrm{f,i}}^H}\mathbf{\tilde{h}_{\mathrm{m,f}}} \mathbf{\tilde{h}}_{\mathrm{m,f}}^H
	\mathbf{u_{\mathrm{f,i}}} \leq I^{Threshold}_{m}
	\end{gather*}
in which the term $ \mathbf{\tilde{h}_{\mathrm{m,f}}} \mathbf{\tilde{h}}_{\mathrm{m,f}}^H$ is always a positive semi-definite matrix and is, therefore, a convex set as shown in 
\cite[p.8,9]{BoV:04}. 
%This is essentially high dimensional ellipsoid.


\item \
	Constraint \eqref{power_const_gen} can be similarly decomposed into the sum
	\begin{gather*}
		\sum_{i=1}^{K_f}\mathbf{u_{\mathrm{f,i}}^{\mathrm{H}}} \mathbf{I} 		
		\mathbf{u_{\mathrm{f,i}}} \leq  P^{Total}_{f}
	\end{gather*}
	in which $\mathbf{I}$ is always positive definite and 			
	therefore the constraint is strictly convex for the same 		
	reason as \eqref{interference_const_gen}.
\end{itemize}

\item 
	Constaint \eqref{zf_const_gen} is an affine constraint. 
		\begin{gather*}
		\langle \mathbf{h_{\mathrm{f,j}}}\mathbf{u_{\mathrm{f,i}}} \rangle =0
		\end{gather*}
%Note that affine constaints to not have to satisfy Slater's condition

\end{enumerate}

\subsection{Potential Game for General Problem}
TODO: 
\begin{itemize}
\item Need to prove that pure strategy NE would exist in this setting.
\item Remove this section??

\end{itemize}

\section{Concave System Model}
An following adaption of the general problem permits a unique Nash Equilibrium to be found via a single convex problem which may be distributed across players of the game.
\subsection{Players: Femtocell Base Stations}


Individual Femtocell base stations (FC-BS) are the players of this game.
\\
Femtocells are characterized by the following parameters
\begin{itemize}
\item 
	FC-BSs with multiple antennas ($T_f >=1$) can beamform their transmission using the precoding 	
	matrix $\mathbf{U}_{\mathrm{f}} \in \mathbb{C}_{T_f \times K_f}$ .
	The columns of $\mathbf{U}_{\mathrm{f}}$ are normalized such that 
	 $\|\mathbf{u}_{\mathrm{fi}}\|^2 =1 \;\forall i \in \{1 ... T_f\}$.
	 $\mathbf{x_{\mathrm{f}}}$ is the 		
	normalized information symbol vector of FC-BS $f$ such that $E[\|\mathbf{x}_{\mathrm{f}}
	\|_2^2]=1$ and $E[\mathbf{x}_{\mathrm{f}}\mathbf{x}_{\mathrm{f}}^H]=\mathbf{I}$ $\forall f \in \{1 ... F\}$.
\\
\item  
	FC-BS $f$ allocates  transmission power using the diagonal, power allocation  	
	matrix $\mathrm{diag}(\mathbf{p}_{\mathrm{f}})$ with $p_{\mathrm{fi}} \geq 0, \forall i \in \{1 ... K_f\}$
such that the transmitted 		
	signal is 
	$\mathbf{s}_{\mathrm{f}	}= \mathbf{U_{\mathrm{f}}} 
	\mathrm{diag}(\mathbf{p}_{\mathrm{f}})
	\mathbf{x_{\mathrm{f}}}$.
\\
\item 
	FC-BS $f$ has power constraint:
	\begin{gather*}
	trace(E[\mathbf{s}_\mathrm{f}\mathbf{s}_\mathrm{f}^H]) =
	 trace(\mathbf{U_{\mathrm{f}}} 
	\mathrm{diag}(\mathbf{p}_{\mathrm{f}})
	E[\mathbf{x_{\mathrm{f}}}
	\mathbf{x_{\mathrm{f}}^H}]
	\mathrm{diag}(\mathbf{p}_{\mathrm{f}})
	\mathbf{U_{\mathrm{f}}}^H 
	)
	  \leq P^{Total}_{f} 
	  	\end{gather*}.

% TODO(Simulate degradation with incomplete CSI solution?)

\item 
$\mathbf{U}_f$ is the normalized, Moore-Penrose inverse to $\mathbf{H_\mathrm{f}}$.
\end{itemize}

\subsection{Macro Cell Users}
No general setup.

\subsection{Femtocell Users}
\begin{itemize}


%\item TODO Decide if there should be minimum rate constraints for Femtocell users in case (Look back at later in case some users are beam-formed out of the transmission). Check if this constraint will disrupt solution.
%\\

\item User $i$ of FC-BS $f$ has SINR:

	\begin{equation*}
	\gamma_{\mathrm{f,i}} = 
	\frac{p_{\mathrm{fi}}\|\mathbf{h^H_{\mathrm{f,i}}u_{\mathrm{f,i}}}x_{\mathrm{fi}}\|^2}
	{\sigma^2_{noise}   +
	\underbrace{
	\sum_{\mathrm{\tilde{f}=1,\tilde{f} \neq f}}^{\mathrm{F}}
	\sum_{\mathrm{\tilde{k}\neq i}}^{\mathrm{K_f}}
	  p_{\mathrm{f\tilde{k}}}\|\mathbf{h^H_{\mathrm{f,\tilde{k}}}u_{\mathrm{f,\tilde{k}}}}x_{\mathrm{f\tilde{k}}}\|^2}_
	  {\mathrm{Inter-cell}}+ \underbrace{
	\sum_{\mathrm{\tilde{k}\neq i}}^{\mathrm{K_f}}
	  p_{\mathrm{f\tilde{k}}}\|\mathbf{h^H_{\mathrm{f,\tilde{k}}}u_{\mathrm{f,\tilde{k}}}}x_{\mathrm{f\tilde{k}}}\|^2}
	 _{\mathrm{Intra-cell}}}
	  \; \mathrm{i \in \{1 ... K_f\}}\end{equation*}
\\
with AWGN $\sim \mathcal{N}(0,\sigma^2_n)$
\\

Assuming negligible inter-cell interference, this reduces to
	\begin{equation*}
	\gamma_{\mathrm{f,i}} = \frac{p_{\mathrm{fi}}\|\mathbf{h^H_{\mathrm{f,i}}
	u_{\mathrm{f,i}}}x_{\mathrm{fi}}\|^2}
	{\sigma^2_{noise} 
	 + \sum_{\mathrm{\tilde{k}\neq i}}^{\mathrm{K_f}}
	  p_{\mathrm{f\tilde{k}}}\|\mathbf{h^H_{\mathrm{f,\tilde{k}}}u_{\mathrm{f,\tilde{k}}}}x_{\mathrm{f\tilde{k}}}\|^2}
	  \; \mathrm{i \in \{1 ... K_f\}}
	\end{equation*}
\\

%This further simplifies assuming that users use a zero-forcing beam-former
%
%\begin{equation}\label{zf_snr}
%\gamma_{\mathrm{f,i}} = \frac{|\mathbf{h^H_{\mathrm{f,i}}u_{\mathrm{f,i}}}|^2}
%{\sigma^2_{noise}  
%}
%\end{equation}
%\\

\end{itemize}


\subsection{Concave Optimization Problem of player $f$}


	\begin{subequations}
	\label{optim}
	\begin{align}
	    \underset{\mathbf{p}_{\mathrm{f}} }{\text{minimize}} \;
	    & - \sum_{\mathrm{i=1}}^{\mathrm{K_f}}
    	U_{\mathrm{f,i}}(p_{\mathrm{f,i}}) \label{player_opt} \\
	    \text{subject to} \; &
	  \sum^F_{f=1} \mathbf{\tilde{h}}_{\mathrm{m,f}}^T  \mathbf{s}_{\mathrm{f}} 						
	\mathbf{s_{\mathrm{f}}^{\mathrm{H}}} \mathbf{\tilde{h}_{\mathrm{m,f}}^*} \leq I^{Threshold}		
	_{\mathrm{m}} & m \in \{1 ...M\} 
		\label{interference_const}\\
        & trace(\mathbf{s}_\mathrm{f}\mathbf{s}_\mathrm{f}^H)  \leq P^{Total}_{f}  \label{power_const}\\
        & p_{\mathrm{f,i}} \geq 0 &  i \in \{1 ...K_{\mathrm{f}}\} \label{pos_power_const}
	\end{align}
	\end{subequations}


\section{Solving the Concave Game}

\subsection{Verifying Concavity of Player Optimization Problem}

Sufficient conditions for a concave problem are:

\begin{enumerate}
\item The utility function is assumed concave in its argument $p_{\mathrm{f,i}}$.

\item
The constraints form a convex, closed and bounded set. 

\begin{itemize}

\item
	Constaint \eqref{interference_const} contains $M$ affine constraints on $diag(\mathbf{p_{\mathrm{f}}})$ and 
	can be rewritten as 

\begin{gather*}
	  \sum^F_{f=1} \mathbf{\tilde{h}}_{\mathrm{m,f}}^T  \mathbf{s}_{\mathrm{f}} 						
	\mathbf{s_{\mathrm{f}}^{\mathrm{H}}} \mathbf{\tilde{h}_{\mathrm{m,f}}^*} = 
	\sum^F_{f=1} trace(\mathbf{\tilde{h}}_{\mathrm{m,f}}^T U_{\mathrm{f}}diag(\mathbf{p_{\mathrm{f}}})^{\frac{1}{2}}\mathbf{x_{\mathrm{f}}}
	 \mathbf{x_{\mathrm{f}}}^H  diag(\mathbf{p_{\mathrm{f}}})^{\frac{1}{2}} U_{\mathrm{f}}^H
	 \mathbf{\tilde{h}}_{\mathrm{m,f}}^*
	)	
	\leq I^{Threshold}_{\mathrm{m}} 
\end{gather*}

\item \
	Constraint \eqref{power_const} is  afine in $diag(\mathbf{p_{\mathrm{f}}})$.
	
\item \
	Constraint \eqref{pos_power_const} is affine in $diag(\mathbf{p_{\mathrm{f}}})$.
\end{itemize}

%Note that affine constaints to not have to satisfy Slater's condition

\end{enumerate}



\subsection{Finding Nash Equilibrium}

\begin{enumerate}
\item \textbf{Existence of Nash Equilibrium:} As the individual, player optimization problem is concave, this is an n-person concave game and therefore a NE exists \cite[Thm1]{rosen1964existence}. 
\item \textbf{Uniqueness of Nash Equilibrium:} In order to use the tools defined in \cite[Thm4]{rosen1964existence} for proving uniqueness of Nash Equilibrium. The function $G(b,r) $ is defined as the Jacobian of $g(b,r) $ which is defined as 

\begin{equation}
g(b,r)= 
\begin{bmatrix}
r_1 \nabla V_{1}(b)
\\
r_2 \nabla V_{1}(b)
\\
\vdots\\
r_F \nabla V_{1}(b)
\end{bmatrix}
\end{equation}

with $r_i>0$.
In the setup of this game, $\nabla V_{1}(b)$
is the gradient of the utility function of FCBS $U_f(\mathbf{U}_{\mathrm{f}}) $ with respect to elements of the  beam-forming matrix 
$\mathbf{U}_{\mathrm{f}}$


\begin{itemize}
\item
Negative Definiteness of the matrix $[G(b,r)+G^{T}(b,r)] $ is a sufficient condition for Diagonally Strict Concavity of the game and therefore implies uniqueness of a NNE in n-person concave games \cite[Thm6]{rosen1964existence}
	 
\item First, \eqref{player_opt} contains no inter-cell interference by assumption and therefore, the derivative with respect to any beam-forming variables from other players is zero. Therefore all off-diagonal elements of $[G(b,r)+G^{T}(b,r)] $ wil be zero.
\item Second, in order to obtain a negative definite result, \eqref{player_opt} must have strictly negative second derivative with respect to the variables in the argument. If we allow
\begin{equation*}
		U_f(\mathbf{U}_\mathrm{f})=
	    - \sum_{\mathrm{i=1}}^{\mathrm{K_f}}
    	U_{\mathrm{f,i}}(\gamma_{\mathrm{f,i}})
\end{equation*}
By definition, $U_{\mathrm{f,i}}()$ is concave in its argument and therefore $- U_{\mathrm{f,i}}()$ is convex.
The argument of the function $- U_{\mathrm{f,i}}()$  is $\gamma_{\mathrm{f,i}}$ which can we expanded under the constraints of  \eqref{optim} can be expanded as 

	\begin{equation}\label{zf_snr_expanded}
	\gamma_{\mathrm{f,i}} = \frac{\|\mathbf{h^H_{\mathrm{f,i}}u_{\mathrm{f,i}}}\|^2}
	{\sigma^2_{noise}  
	}
	= 
	\frac{\mathbf{u^H_{\mathrm{f,i}}h_{\mathrm{f,i}}h^H_{\mathrm{f,i}}u_{\mathrm{f,i}}}}
	{\sigma^2_{noise}  
	}
	\end{equation}
	
	Noting that the matrix 
	$\mathbf{h}_{\mathrm{f,i}}\mathbf{hh}^H_{\mathrm{f,i}}$
	is limited to rank = 1. This is only a positive semidefinite function in 
	$u_{\mathrm{f,i}}$ and therefore is convex but not strictly convex.


\end{itemize}


\item
Summary of above: 
A unique NE exist for every $\mathbf{r} \geq 0$ 

\end{enumerate}

\subsection{Potential Game for Concave Problem}

\subsubsection{Setup as a Potential Game}
It is useful to now represent the game by a "Potential Function" 
$ \Psi(\mathbf{})$ which satisfies:
\begin{equation}\label{potential_game_condition}
\frac{d\Psi}{du_\mathrm{f,i}} = \frac{d U_f()}{du_\mathrm{f,i}} \; \forall f \in \{1...F\}
\end{equation}.


With this condition satisfied, the solution to the optimization problem 

	
		\begin{subequations}
	\label{optim}
	\begin{align}
	    \underset{\mathbf{p}}{\text{minimize}}
	    & \; \Psi(\mathbf{p}) \label{potential_game} \\
	    \text{subject to} \; &
	  \sum^F_{f=1} \mathbf{\tilde{h}}_{\mathrm{m,f}}^T  \mathbf{s}_{\mathrm{f}} 						
	\mathbf{s_{\mathrm{f}}^{\mathrm{H}}} \mathbf{\tilde{h}_{\mathrm{m,f}}^*} \leq I^{Threshold}		
	_{\mathrm{m}} & m \in \{1 ...M\} 
		\label{interference_const}\\
        & trace(\mathbf{s}_\mathrm{f}\mathbf{s}_\mathrm{f}^H)  \leq P^{Total}_{f}  \label{power_const}
        & \forall f \in \{1 ... F\}\\
        & p_{\mathrm{f,i}} \geq 0 &  \forall i \in \{1 ...K_{\mathrm{f}}\} \forall f \in \{1 ... F\}\label{pos_power_const}
	\end{align}
	\end{subequations}

is a normalized Nash Equilibrium of the original problem.
As shown above, this Nash Equilibrium exists and is unique. 
\\
\textbf{Potential Function:} The proposed potential function is 

\begin{gather*} \label{Potential_Function}
\Psi() = \sum_{f = 1}^{F} U_f() 
\end{gather*}

This is the primary motivation for enforcing $U_f$ as the zero-forcing precoder as this allows the game to satisfy \eqref{potential_game_condition}. 

%\begin{theorem}\label{distributed}
%\cite{ghosh2015normalized}
%If a game's potential function is strictly concave and the derivative of the function with respect to the individual players variables are independent of the other player variables, then there exists a distributed solution.
%\end{theorem}

\subsection{Distributed Solution to the Game}
A desirable feature to \eqref{potential_game} is for methods for reaching the solution to be distributable. This may allow for minimal communication overhead between processes (in this case players and macro users).
\subsubsection{Central Problem Resulting from Potential Game}
As described in \cite[p.~8,9]{boyd2011distributed}, the dual ascent method can be used to find a distributed solution to this problem using the Lagrangian of \eqref{potential_game}. 
\\
\begin{multline}
L(\mathbf{U,\lambda}) = 
\;
\sum_{f=1}^F U_f() 
+
\sum_{\mathrm{m=1}}^M \lambda_{\mathrm{m}}
(	  \sum^F_{f=1} \mathbf{\tilde{h}}_{\mathrm{m,f}}^T  \mathbf{s}_{\mathrm{f}} 						
	\mathbf{s_{\mathrm{f}}^{\mathrm{H}}} \mathbf{\tilde{h}_{\mathrm{m,f}}^*} - I^{Threshold}		
	_{\mathrm{m}} )
\\
+ 
\sum_{f=1}^F
\lambda_{\mathrm{f}}^{'}(trace(\mathbf{s}_\mathrm{f}\mathbf{s}_\mathrm{f}^H)-P^{Total}_{f} )
+
\sum_{f=1}^F
\sum_{i=1}^{K_f}
\nu_{\mathrm{f,i,j}}(-p_{\mathrm{f,i}})
\end{multline}

The corresponding dual function and dual problem are then 
\begin{gather*}
g(\lambda,\nu) = \underset{\mathbf{U}}{\mathrm{argmin}}\;L(\mathbf{U,\lambda})
\end{gather*}
\begin{gather*}
g(\lambda,\nu) = \underset{\lambda}{\mathrm{argmax}}\;\underset{\mathbf{U}}{\mathrm{argmin}}\;L(\mathbf{U,\lambda})
\end{gather*}
.



This dual function can then be decomposed into F component functions
\begin{multline}
g_f(\lambda,\nu) = \underset{\mathbf{p_f}}{\mathrm{argmin}}
\{
\;
U_f() 
+
\sum_{\mathrm{m=1}}^M \lambda_{\mathrm{m}}
(\mathbf{\tilde{h}}_{\mathrm{m,f}}^T  \mathbf{s}_{\mathrm{f}} 						
	\mathbf{s_{\mathrm{f}}^{\mathrm{H}}} \mathbf{\tilde{h}_{\mathrm{m,f}}^*} - I^{Threshold}		
	_{\mathrm{m}} )
\\
+ 
\lambda_{\mathrm{f}}^{'}(trace(\mathbf{s}_\mathrm{f}\mathbf{s}_\mathrm{f}^H)-P^{Total}_{f} )
+
\sum_{i=1}^{K_f}
\nu_{\mathrm{f,i}}(-p_{\mathrm{f,i}})\}
\end{multline}
\\

The following steps can then be iterated in order to reach an optimal solution. 
\begin{enumerate}
\item 
Individual players can solve $ g_f(\lambda,\nu) $ independently.
TODO: The Lagrangian is only guaranteed to be convex wrt the dual variables thus there is no guarantee this problem is bounded. 
\item 
Using $g(\lambda,\nu) = \sum_{f=1}^{F}g_f(\lambda,\nu)$ and the calculus of subgradients $\partial g(\lambda,\nu) = \sum_{f=1}^{F} \partial g_f(\lambda,\nu)$, the dual variables can updated by 

\begin{gather}
\lambda_{\mathrm{m}}^{\mathrm{k+1}} = 
\lambda_{\mathrm{m}}^{\mathrm{k}}
+
\alpha^{\mathrm{k}}*
\partial g(\lambda,\nu)
\end{gather}

using to predefined $\alpha^{\mathrm{k}}$ which must satisfy certain sumability conditions.



\end{enumerate} 

TODO 
\begin{itemize}
\item show proof convergence conditions of this algorithm
\item check if the distributed solution depends on strict convexity (or is  the solution for these methods needs to be unique)-> Doesn't seem so
\item see if 1st update is bounded (if not, a new method will be needed)
\item begin looking at how to choose the update step-size in the second step
\item after writing explicitly, verify that no information passing between Femtocells will be needed 
\end{itemize}

\chapter{Results}

\chapter{Conclusion}

\newpage
\bibliography{gt_report}
\end{document}