
\documentclass[12pt]{article}
\usepackage{fullpage,graphicx,psfrag,amsmath,amsfonts,verbatim}
\usepackage[small,bf]{caption}

\input defs.tex

\bibliographystyle{alpha}

\title{Power Allocation in Heterogeneous Networks with Multiple Antenna }
\author{Peter Hartig}


\newtheorem{theorem}{Theorem}


\begin{document}
\maketitle


%\begin{abstract}
This game considers a wireless communications network which includes both macro cell and femto cell users. In order to allow for uncoordinated usage of femto cells within the network, femto cell base stations must ensure that the users served by macro cells only intercept an acceptable amount of interference during transmission. The strategy for transmission across all such femto cells in a network is investigated here in a game theoretic context. 

%\end{abstract}

\newpage
\tableofcontents
\newpage

\section{Game and System Setup}

\subsection{Game Elements}

\subsubsection{Players: Femto Cell Base Stations}


Individual femto cell base stations (FC-BS) are the players of this game.
\\
Femto Cells are characterized by the following parameters
\begin{itemize}
\item 
	Each FC-BS  $f \in \{1 ... F\}$ is considered to have a number of antennas $T_f$ with which to transmit to $K_f$ femto cell users. It is assumed throughout the remainder that $T_f \geq K_f$.
\\
\item 
	FC-BSs with multiple antennas ($T_f >1$) can beamform their transmission using the precoding 	
	matrix $\mathbf{U}_{\mathrm{f}} \in \mathbb{C}_{T_f \times K_f}$ such that the transmitted 		
	signal is $\mathbf{s}_{\mathrm{f}
	}= \mathbf{U_{\mathrm{f}}}\mathbf{x_{\mathrm{f}}}$. Here $\mathbf{x_{\mathrm{f}}}$ is the 		
	normalized vector of symbols for users of FC-BS $f$ (i.e. $E[\|\mathbf{x}_{\mathrm{f}}
	\|_2^2] \; f \in \{1 ... F\}$).
\\
\item 
	FC-BS $f$ has power constraint $trace(\mathbf{U}_f^H\mathbf{U}_f) \leq P^{Total}_{f} $.
\\
\item
	 FC-BSs are assumed to be spaced far apart in distance. Therefore, FC-BS $f$ can be modeled as 
	 causing no interference to the users of FC-BS $j \in \{1 ... F\}\backslash f$
\item 
	FC-BSs are assumed to have a utility function $U_f()$ based upon the quality of service 		
	provided to its users. TODO (Discuss Reasonable functions)
\\
\item 
	FC-BS $f$ is assumed to know the downlink channel ($\mathbf{H_\mathrm{f}}$) from its transmission 		
	antennas to all served users.
% TODO(Simulate degradation with incomplete CSI solution?)
\\
\end{itemize}

\subsubsection{Macro Cell Users}

\begin{itemize}
\item 
	Macro Cell user $m \in \{1 ... M\}$ experiences receiver interference due to transmission by
	FC-	BSs. These macro-cell users have limits to the amount of interference they may tolerate 
	$\sum^F_{f=1} \mathbf{\tilde{h}}_{\mathrm{m,f}}^T  \mathbf{U_{\mathrm{f}}} 						
	\mathbf{U_{\mathrm{f}}^{\mathrm{H}}} \mathbf{\tilde{h}_{\mathrm{m,f}}^*} \leq I^{Threshold}		
	_{\mathrm{m}} $.

\item 
	FC-BS $f$ is assumed to know the downlink channel ($\tilde{\mathbf{H}_{\mathrm{f}}}$) from its $T_f$
	transmission antennas to all $M$ macro-cells with which it interferes.
\\
\end{itemize}

\subsubsection{Femto Cell Users}
\begin{itemize}


%\item TODO Decide if there should be minimum rate constraints for femto cell users in case (Look back at later in case some users are beam-formed out of the transmission). Check if this constraint will disrupt solution.
%\\

\item Users of FC-BS $f$ have SINR:
	\begin{equation*}
	\gamma_{\mathrm{f,i}} = \frac{\|\mathbf{h^H_{\mathrm{f,i}}u_{\mathrm{f,i}}}\|^2}
	{\sigma^2_{noise}   +
	\underbrace{
	 \sum_{\mathrm{\tilde{f}}\neq f} \sum_{\mathrm{u=1}}^{K_{\mathrm{\tilde{f}}}}
	\|\mathbf{h^H_{\mathrm{\tilde{f},u}}u_{\mathrm{f,i}}}\|^2}_{\mathrm{Inter-cell}}
	 + 
	 \underbrace{
	 \sum_{\mathrm{\tilde{k}\neq i}}
	 \|\mathbf{h^H_{\mathrm{f,\tilde{k}}}u_{\mathrm{f,\tilde{k}}}}\|^2}_{\mathrm{Intra-cell}}}
	  \; \mathrm{i \in \{1 ... K_f\}}\end{equation*}
\\
with AWGN $\sim \mathcal{N}(0,\sigma^2_n)$
\\

Assuming negligible inter-cell interference, this reduces to
	\begin{equation*}
	\gamma_{\mathrm{f,i}} = \frac{\|\mathbf{h^H_{\mathrm{f,i}}u_{\mathrm{f,i}}}\|^2}
	{\sigma^2_{noise} 
	 + \sum_{\mathrm{\tilde{k}\neq i}}
	  \|\mathbf{h^H_{\mathrm{f,\tilde{k}}}u_{\mathrm{f,\tilde{k}}}}\|^2}
	  \; \mathrm{i \in \{1 ... K_f\}}
	\end{equation*}
\\

%This further simplifies assuming that users use a zero-forcing beam-former
%
%\begin{equation}\label{zf_snr}
%\gamma_{\mathrm{f,i}} = \frac{|\mathbf{h^H_{\mathrm{f,i}}u_{\mathrm{f,i}}}|^2}
%{\sigma^2_{noise}  
%}
%\end{equation}
%\\

\end{itemize}


%\subsection{Variations of Game Setup}
%
%\subsubsection{Case: $T_f \geq M + K_f$}
%FC-BSs could potentially zero-beam-form towards all macro users. However, as base stations have power constraints, it may be beneficial to cause certain amounts of interference. 
%
%\subsubsection{Case: $K_f \leq T_f < M + K_f$}
%FC-BSs can send unique signals to all users but does not have sufficient DOF to zero-beam-form for all macro users.


\subsection{General Optimization Problem}

Each player $f$ attempts to maximize utility function $U_f()$ while playing a strategy that falls in the region constrained by the interference constraints imposed by the macro cell users.
\par

If intra-cell interference is  prohibited by the restriction of $\mathbf{U}_f$ to the set of zero-forcing matrices, the player optimization problem of player $f$ can be written as:
\\\\
TODO: Clarify why the player utility function should be broken out like this. This would not allow for intracell interference later on. 

	
	\begin{subequations}
	\label{optim}
	\begin{align}
	    \underset{\mathbf{U}_{\mathrm{f}} }{\text{minimize}} \;
	    & - \sum_{\mathrm{i=1}}^{\mathrm{K_f}}
    	U_{\mathrm{f,i}}(\gamma_{\mathrm{f,i}}) \label{player_opt} \\
	    \text{subject to} \; &
	   \sum^F_{f=1} \mathbf{\tilde{h}}_{\mathrm{m,f}}^T  \mathbf{U_{\mathrm{f}}}		
	\mathbf{U_{\mathrm{f}}^{\mathrm{H}}} \mathbf{\tilde{h}_{\mathrm{m,f}}^*} \leq I^{Threshold}		
	_{\mathrm{m}} & m \in \{1 ...M\} 
		\label{interference_const}\\
        & trace(\mathbf{U_f^H}\mathbf{U_f}) \leq P^{Total}_{f} \label{power_const}\\
        & \langle \mathbf{h_{f,j}}\mathbf{u_{f,i}} \rangle =0\ & \; \forall j \in \{1... K_f\}\backslash i ,\; \forall i \in \{1 ... K_f\} \label{zf_const}
	\end{align}
	\end{subequations}

	
Note that over the feasible region of this problem, the SINR of femto cell users reduces to 

	\begin{equation}\label{zf_snr}
	\gamma_{\mathrm{f,i}} = \frac{\|\mathbf{h^H_{\mathrm{f,i}}u_{\mathrm{f,i}}}\|^2}
	{\sigma^2_{noise}  
	}
	= 
	\frac{\mathbf{u^H_{\mathrm{f,i}}h_{\mathrm{f,i}}h^H_{\mathrm{f,i}}u_{\mathrm{f,i}}}}
	{\sigma^2_{noise}  
	}
	\end{equation}
due to  \eqref{zf_const}
\section{Solving the Game}
\subsection{Player Utility Functions}

Say $U_f() $ is the sum of all capacities for femto cell users of FC-BS $f$ 
\begin{equation*}
U_f(\mathbf{U}_\mathrm{f}) = \sum^{K_f}_{i=1} log(1+\gamma_{\mathrm{f,i}})
\end{equation*}
TODO generalize to a set of functions that satisfy required concavity conditions
\begin{itemize}
\item needs to satisfy condition of creating an n-persion concave game
\item needs to fulfill conditions in order to create negative definite $[G(b,r)+G^{T}(b,r)] $
\end{itemize}

\subsection{Verifying Conexity of Player Optimization Problem}

Sufficient conditions for a convex problem are:

\begin{enumerate}
\item The utility function is concave in its argument 
\begin{itemize}
\item 
First note that constraint \eqref{zf_const}  ensures that $\gamma_{\mathrm{f,i}}$ takes the form of \eqref{zf_snr}.
As $U_f() $ is concave in its argument by definition and the argument is a positive semi-definite quadratic term in the variables, the result is concave.
\end{itemize}

\item
Constraints form convex, closed and bounded set. 
\\
TODO show closed and boundedness of set

\begin{itemize}

\item
	Constaint \eqref{interference_const} contains $M$ quadratic constraints on $\mathbf{U_f}$ and 
	can be rewritten as 

\begin{gather*}
	\sum_{f=1}^F
	trace(\mathbf{U_f^H} \mathbf{\tilde{h}_{m,f}} \mathbf{\tilde{h}_{m,f}^H} \mathbf{U_f} )\leq 
	I^{Threshold}_{m}.
\end{gather*}
This can be decomposed into \textit{independent} components 
	\begin{gather*}
	\sum_{f=1}^F
	\sum_{i=1}^{f_i}
	\mathbf{u_{\mathrm{f,i}}^H}\mathbf{\tilde{h}_{\mathrm{m,f}}} \mathbf{\tilde{h}}_{\mathrm{m,f}}^H
	\mathbf{u_{\mathrm{f,i}}} \leq I^{Threshold}_{m}
	\end{gather*}
in which the term $ \mathbf{\tilde{h}_{\mathrm{m,f}}} \mathbf{\tilde{h}}_{\mathrm{m,f}}^H$ is always a positive semi-definite matrix and is, therefore, a convex set as shown in 
\cite[p.8,9]{BoV:04}. 
%This is essentially high dimensional ellipsoid.


\item \
	Constraint \eqref{power_const} can be similarly decomposed into the sum
	\begin{gather*}
		\sum_{i=1}^{K_f}\mathbf{u_{\mathrm{f,i}}^{\mathrm{H}}} \mathbf{I} 		
		\mathbf{u_{\mathrm{f,i}}} \leq  P^{Total}_{f}
	\end{gather*}
	in which $\mathbf{I}$ is always positive definite and 			
	therefore the constraint is strictly convex by the same 		
	reasons as \eqref{interference_const}.
\end{itemize}

\item 
	Constaint \eqref{zf_const} is an affine constraint. 

		\begin{gather*}
		\langle \mathbf{h_{\mathrm{f,j}}}\mathbf{u_{\mathrm{f,i}}} \rangle =0
		\end{gather*}
%Note that affine constaints to not have to satisfy Slater's condition

\end{enumerate}



\subsection{Finding Nash Equilibrium}

\begin{enumerate}
\item \textbf{Existence of Nash Equilibrium:} Given that the player optimization problem is convex, this is an n-person concave game and therefore a NE exists \cite[Thm1]{rosen1964existence}. 
\item \textbf{Uniqueness of Nash Equilibrium:} In order to use the tools defined in \cite[Thm4]{rosen1964existence} for proving uniqueness of Nash Equilibrium. The function $G(b,r) $ is defined as the Jacobian of $g(b,r) $ which is defined as 

\begin{equation}
g(b,r)= 
\begin{bmatrix}
r_1 \nabla V_{1}(b)
\\
r_2 \nabla V_{1}(b)
\\
\vdots\\
r_F \nabla V_{1}(b)
\end{bmatrix}
\end{equation}

with $r_i>0$.
In the setup of this game, $\nabla V_{1}(b)$
is the gradient of the utility function of FCBS $U_f(\mathbf{U}_{\mathrm{f}}) $ with respect to elements of the  beam-forming matrix 
$\mathbf{U}_{\mathrm{f}}$


\begin{itemize}
\item
Negative Definiteness of the matrix $[G(b,r)+G^{T}(b,r)] $ is a sufficient condition for Diagonally Strict Concavity of the game and therefore implies uniqueness of a NNE in n-person concave games \cite[Thm6]{rosen1964existence}
	 
\item First, \eqref{player_opt} contains no inter-cell interference by assumption and therefore, the derivative with respect to any beam-forming variables from other players is zero. Therefore all off-diagonal elements of $[G(b,r)+G^{T}(b,r)] $ wil be zero.
\item Second, in order to obtain a negative definite result, \eqref{player_opt} must have strictly negative second derivative with respect to the variables in the argument. If we allow
\begin{equation*}
		U_f(\mathbf{U}_\mathrm{f})=
	    - \sum_{\mathrm{i=1}}^{\mathrm{K_f}}
    	U_{\mathrm{f,i}}(\gamma_{\mathrm{f,i}})
\end{equation*}
By definition, $U_{\mathrm{f,i}}()$ is concave in its argument and therefore $- U_{\mathrm{f,i}}()$ is convex.
The argument of the function $- U_{\mathrm{f,i}}()$  is $\gamma_{\mathrm{f,i}}$ which can we expanded under the constraints of  \eqref{optim} can be expanded as 

	\begin{equation}\label{zf_snr_expanded}
	\gamma_{\mathrm{f,i}} = \frac{\|\mathbf{h^H_{\mathrm{f,i}}u_{\mathrm{f,i}}}\|^2}
	{\sigma^2_{noise}  
	}
	= 
	\frac{\mathbf{u^H_{\mathrm{f,i}}h_{\mathrm{f,i}}h^H_{\mathrm{f,i}}u_{\mathrm{f,i}}}}
	{\sigma^2_{noise}  
	}
	\end{equation}
	
	Noting that the matrix 
	$\mathbf{h}_{\mathrm{f,i}}\mathbf{hh}^H_{\mathrm{f,i}}$
	is limited to rank = 1. This is only a positive semidefinite function in 
	$u_{\mathrm{f,i}}$ and therefore is convex but not strictly convex.

\end{itemize}


\item
Summary of above: 
While NE for the game are known to exist for the problem, these are not necessarily unique. 

\end{enumerate}

%\end{document}




\subsection{Setup as a Potential Game}
It is useful to now represent the game as a "Potential Game". This is defined by a function
$ \Psi(\mathbf{U})$ which satisfies the condition
\begin{equation}\label{potential_game_condition}
\frac{d\Psi}{du_f} = \frac{d U()_f}{du_f}
\end{equation}


With this condition satisfied, the solution to the maximization problem 

\begin{subequations}
	\label{optim}
	\begin{align}
	    \underset{\mathbf{U}}{\text{minimize}}
	    & \; \Psi(\mathbf{U}) \label{potential_game} 
	    \\
	    \text{subject to}  & \;
	    \sum^F_{f=1}\mathbf{\tilde{h_{m,f}^T}}  \mathbf{U_f}  \mathbf{U_f^H}
		\mathbf{\tilde{h_{m,f}^*}} \leq I^{\mathrm{Threshold}}_{m} & m \in \{1 ...M\} 
		\label{interference_const_central}\\
        & trace(\mathbf{U_f^H}\mathbf{U_f}) \leq P^{Total}_{f} & \forall f \in \{1 ... F\}
        \label{power_const_central}\\
        & \langle \mathbf{h_{f,j}}\mathbf{u_{f,i}} \rangle =0\ \; \forall j \in \{1... K_f\}				\backslash i &\forall i \in \{1 ... K_f\}\; \forall f \in \{1 ... F\} \label{zf_const_central}
	\end{align}
	\end{subequations}

is a normalized Nash Equilibrium of the original problem.
As it has been above, this Nash Equilibrium to this problem exists (though not necessarily unique). 
\\
\textbf{Potential Function:} The proposed potential function is 

\begin{gather*} \label{Potential_Function}
\Psi() = \sum_{f = 1}^{F} U_f() 
\end{gather*}


As the utility functions of individual players $f$ do not share any common variables,  \label{Potential_Function} will have the same derivative as individual $U_f()$ with respect to $\mathbf{U_{f,i}} $, satisfying condition \eqref{potential_game_condition}. 

%\begin{theorem}\label{distributed}
%\cite{ghosh2015normalized}
%If a game's potential function is strictly concave and the derivative of the function with respect to the individual players variables are independent of the other player variables, then there exists a distributed solution.
%\end{theorem}

\subsection{Distributed Solution to the Game}
\subsubsection{Central Problem Resulting from Potential Game}
As described in \cite[p.~8,9]{boyd2011distributed}, the dual ascent method can be used to find a distributed solution to this problem using the Lagrangian of \eqref{potential_game}
\\
\begin{multline}
L(\mathbf{U,\lambda}) = 
\;
\sum_{f=1}^F
\sum_{i=1}^{K_f}
log(1+\frac{p_{ \mathrm{f,i}}|\mathbf{h^H_{\mathrm{fi}}u_{ \mathrm{fi}}}|^2}{\sigma^2_{ \mathrm{noise}} })
+
\sum_{\mathrm{m=1}}^M \lambda_{\mathrm{m}}
(\sum_{\mathrm{f=1}}^F
\sum_{\mathrm{i=1}}^{K_f}
\mathbf{u_{ \mathrm{f,i}}^H} \mathbf{\tilde{h_{m,f}}} \mathbf{\tilde{h_{m,f}^H}} \mathbf{u_{\mathrm{f,i}}} - I^{Threshold}_{m} )
\\
+ 
\sum_{f=1}^F
\lambda_{\mathrm{f}}^{'}(
\sum_{i=1}^{K_f}\mathbf{u_{f,i}^H} \mathbf{I} \mathbf{u_{\mathrm{f,i}}} -  P^{Total}_{f})
+
\sum_{f=1}^F
\sum_{i=1}^{K_f}
\sum_{j=1, j\neq i}^{K_f}
\
\nu_{\mathrm{f,i,j}}(\mathbf{h^T_{\mathrm{f,j}}}\mathbf{u_{\mathrm{f,i}}})
\end{multline}

in which individual FC-BSs first individually optimize 
\begin{enumerate}
\item 
\begin{gather}
\mathbf{U^{\mathrm{k+1}}_{\mathrm{f}}} =\mathrm{argmax}_{\mathbf{U_{\mathrm{f}}}} \; L(\mathbf{U,\lambda})
\end{gather}
and then macro stations will take a gradient ascent step using a 
predetermined $\alpha^{\mathrm{k}}$
\item 
\begin{gather}
\lambda_{\mathrm{m}}^{\mathrm{k+1}} = 
\lambda_{\mathrm{m}}^{\mathrm{k}}
+
\alpha^{\mathrm{k}}*
(\frac{\partial L(\mathbf{\mathbf{U},\lambda}) )}{\partial\lambda_{\mathrm{m}}})
\end{gather}
in which $\mathbf{U^{\mathrm{k}}} = [\mathbf{U^{\mathrm{k}}_{\mathrm{1}}}...\mathbf{U^{\mathrm{k}}_{\mathrm{F}}}]$

\end{enumerate} 

TODO 
\begin{itemize}
\item show proof convergence conditions of this algorithm
\item check if the distributed solution depends on strict convexity (or is  the solution for these methods needs to be unique)-> Doesn't seem so
\item see if 1st update is bounded (if not, a new method will be needed)
\item begin looking at how to choose the update step-size in the second step
\item after writing explicitly, verify that no information passing between femto cells will be needed 
\end{itemize}

\section{Uniqueness of Zero-Forcing}
Also of interest is studying whether or not the zero-forcing matrix typically chosen which minimizes the error covariane matrix is unique in the presence of interference constraints. 

\newpage
\bibliography{system_model_bib}

\end{document}
